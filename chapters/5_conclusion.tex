\section{Conclusion}
\label{sec:conclusion}

Les modèles de la famille \ARIMA permettent de modéliser certains phénomènes variant avec le temps comme ici des ventes, et de faire des prévisions pour les valeurs futures du phénomène. Ce sont des outils qui fonctionnent le plus souvent rapidement et avec peu d'hypothèse à vérifier. Ici malgré les particularités des données, avec une bonne prise en main ce sont des outils utiles et efficaces.

Bien sûr il a été difficile d'obtenir des résultats dès le début. Un gros travail était nécessaire pour prétraiter et analyser les données. Les modèles \ARIMA et la prédiction de série temporelle en général ne sont pas facile à appréhender. N'ayant jamais eu de cours ni de précédents avec ce domaine, nous étions perdu. Désireux d'obtenir rapidement des résultats et en appliquant ces modèles comme des outils magiques qui fonctionnent du premier coup, nous avons longtemps été dans l'incapacité d'avancer ou même d'utiliser une démarche cohérente. Nous nous sommes lancés dans la recherche d'optimisation de nos modèles \ARIMA avec de longues recherches de paramètres ou d'ajout de variables explicatives, sans réel succès.

Avec l'aide de notre professeur encadrant, nous nous sommes réorientés vers une réflexion sur l'organisation des données, c'est à dire sur comment palier aux particularité des données et traiter les semestres de façon similaire. Cette réflexion n'as pas aboutit tout de suite et nous avons dû réfléchir à différentes solutions, plus ou moins fructueux, dont celles présentées dans le rapport. Au final nous sommes assez satisfaits des résultats obtenus bien qu'ils ne soient sûrement pas optimaux et que nous n'avons travaillé qu'avec une petite quantité de produits différents.

De plus, la création d'un site permettant la visualisation des données et des différents modèles est une étape supplémentaire vers la mise en production d'un tel projet.

Malgré la difficulté du projet, notre recherche nous aura dans un premier temps permit de découvrir un domaine qui nous était inconnu. Nous avons appris à force de temps les différentes étapes de réflexions autour des séries temporelles, mais aussi les limites de ces modèles. Enfin l'état du projet DataPic a progressé. Les prédictions semblent correctes et sont facilement intégrables dans l'interface développée, ce qui nous rapproche de la mise en production d'une interface ergonomique complète pour la prédiction des ventes au Pic'Asso.

\bigskip

\center{\large Merci de votre lecture}
